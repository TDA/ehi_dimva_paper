\subsection{System Components}
\label{Comp}

The Data Gathering module and Payload Injection modules are composed of smaller components. This section describes the functionality of those components.

\subsubsection{Data Gathering}
\label{Comp:Crawler}
We used an open-source Apache Nutch based Crawler~\cite{nutch}. The \texttt{Crawler} provides the system with a continuous feed of URLs and the HTML contained in those pages. 
\label{Comp:FP}
The \texttt{Form Parser} is responsible for parsing the HTML and retrieving data about the HTML
forms on the page, including the following: (1) Form attributes, such
as \texttt{method} and \texttt{action} (URL) for the HTTP request, (2)
Data about form inputs, such as their attributes, names, and
default values. The default values are essential for fields like
{\lstinline{<input type="hidden">}} as these fields are usually used
to check for the submission of forms by bots, and (3) Presence of the {\lstinline{<base>}} element in the HTML, as this affects the final URL to which the form is to be submitted (if the \texttt{action} attribute is a relative URL).

\label{Comp:EMFC}
The \texttt{\Email Field Checker} is the final stage in the Data Gathering module. It receives the HTML form data and checks for the presence of \email fields in those forms. If any \email fields are found, it stores references to these forms.
The intuition here is that we do not want to try to fuzz all HTML forms on the web to look for \ehi vulnerabilities, rather just those HTML forms that are likely to invoke server-side email functionality.

The \Email Field Checker searches for the words \texttt{e-mail},
\texttt{mail} or \texttt{email} within the form, instead of an
explicit HTML5 \email field (e.g.,\ {\lstinline{<input type="email">}}). This is by design, taking into account a common
design pattern used by web developers, where they may have a text
field with an \texttt{id} or \texttt{name} attribute set to
\texttt{email}, instead of an actual \email type attribute, for
purposes of backward compatibility with older browsers. The output of this stage is stored in the database and acts as the input to the Payload Injection module.

%% Compared to searching for explicit \email fields, by searching for the presence of the words \texttt{\email}, \texttt{mail} or \texttt{email} in the form, we are assured very few false negatives. This is because our system is bound to find \email fields with their \texttt{type}, \texttt{name}, or \texttt{id} set to one of these words. The system is also substantially faster as we do not have to parse the individual form fields at this point in the pipeline. However, despite the advantages, this might also lead to a false positive rate. We discuss this possibility in detail in Section~\ref*{issues:fpr} - Design Issues.


\subsubsection{\Email Form Retriever}
\label{Comp:EMFR}
The \Email Form Retriever is the first stage in the Payload Injection
module. It does the following: (1) Retrieve forms and
remove any duplicates, (2) reconstruct
each form's input fields
and values with the stored form data, and (3) construct the target URL
to create an HTTP request for fuzzing.


\subsubsection{Fuzzer}
\label{Comp:Fuzzer}
% Adam: we should call these something else but modules. We already have the top two modules, which are composed of components, now we need to split these into something (not components again). - DONE
The Fuzzer interacts directly with the external web applications. The system injects payloads in two stages: the goal is to reduce the total number of HTTP requests the system generates to detect an \ehi vulnerability. Making HTTP requests is an expensive process~\cite{httpperf}, and can cause bottlenecks in a Crawler-Fuzzer system~\cite{ShkapenyukTorstenSuel2001}.
The two different types of payloads used for fuzzing are:

\noindent\textbf{Non-Malicious Payload.}
The non-malicious payload is simply an \email address. The goal is to see if the web application will send an \email message based on our input. The specific format of the \email is \lstinline|reguser#@example.com|, where \texttt{\#} is replaced by an internal id that uniquely maps the payload to the form, and \texttt{example.com} is replaced by our domain.

\noindent\textbf{Malicious Payload.}
After receiving an \email from a specific form, we use the malicious payload to try to exploit an \ehi vulnerability. We inject the form fields with the \texttt{bcc} (blind carbon copy) header. If the vulnerability is present, this will cause the server to send a copy of the \email to the \email address we added as the \texttt{bcc} field.

We consider a special case: the addition of an \texttt{x-check:in} header field to the payloads. This is due to Python's exhibited behavior when attaching
headers. Instead of overwriting a header if it is already present, Python will ignore duplicate headers. So, if the \texttt{bcc} field is already present as part of the headers, our injected \texttt{bcc} header would be ignored. To overcome this, we inject a new header that is not likely to be generated by the web application. 

We created four different malicious payloads. Each of these payloads
is crafted for a particular use case. The four payloads are:
(1) nuser\#@example.com\textbackslash{}n\-bcc:\-maluser\-\#\-@example.com,
(2) nuser\#@\-example.com\textbackslash{}n\-bcc:\-maluser\-\#\-@example.com\textbackslash{}n\-x-check:in,
(3) nuser\#@\-example.com\textbackslash{}r\textbackslash{}n\-bcc:\-maluser\#\-@example.com,
and (4) nuser\#\-@example.com\-\textbackslash{}r\textbackslash{}n\-bcc:\-maluser\#\-@example.\-com\textbackslash{}r\textbackslash{}n\-x-check:in.
	
Payload 1 is the most minimal payload: it injects a newline character followed by the \texttt{bcc} field. Payload 2 contains the additional \texttt{x-check} header to inject Python-based web applications. Payloads 3 and 4 are added for purposes of cross-platform fuzzing: \texttt{\textbackslash{}r\textbackslash{}n} is the ``Carriage Return - New Line (CRLF)'' used on Windows systems~\cite{rfc2616}. The \# are replaced by an internal id, for mapping to the forms.

% Adam: If we need to cut things for space, we can cut this next line and the payload coverage table.

%\begin{table}[tbp]
%	\centering
%	\normalsize
%	\begin{tabular}{|c|c|c|}
%		\hline
%		\multicolumn{1}{|c|}{\textbf{Payload}} & \multicolumn{1}{c}{\textbf{Languages covered}} & \multicolumn{1}{|c|}{\textbf{Platforms covered}}\\
%		\hline
%		1 & PHP, Java, Ruby, etc. & Unix\\
%		\hline
%		2 & PHP, Java, Ruby, etc. & Windows\\
%		\hline
%		3 & Python & Unix\\
%		\hline
%		4 & Python & Windows\\
%		\hline
%	\end{tabular}
%	\caption[\titlecap{Payload coverage}]{Payload coverage, each payload covers a different platform/language.}
%	\label{tab:payloadcov}
%\end{table}

%% \begin{table}[tbp]
%% 	\centering
%% 	\scriptsize
%% 	\begin{tabular}{|c|c|c|}
%% 		\hline
%% 		\multicolumn{1}{|c|}{\textbf{Language/Platform}} & \multicolumn{1}{c}{\textbf{Unix}} & \multicolumn{1}{|c|}{\textbf{Windows}}\\
%% 		\hline
%% 		\textbf{PHP, Java, Ruby} & 1 & 3\\
%% 		\hline
%% 		\textbf{Python} & 2 & 4\\
%% 		\hline
%% 	\end{tabular}
%% 	\caption[\titlecap{Payload coverage}]{Payload coverage, each payload covers a different platform/language.}
%% 	\label{tab:payloadcov}
%% \end{table}

Along with the payload, the Fuzzer also injects data into the other fields of the form. This data must pass validation constraints on the individual input fields (e.g.,\ name field might not be allow numbers).  As crawling and fuzzing input fields on the web is an open problem~\cite{raghavan2000crawling}, we chose to go with a best-effort approach. To maximize the amount of vulnerabilities the system discovers, the data injected into the input fields should adhere to the constraints. The Fuzzer uses a ``Data Dictionary'' which has predefined ``keys'' and ``values'' for standard input fields such as \texttt{name}, \texttt{date}, \texttt{username}, \texttt{password}, \texttt{text}, and \texttt{submit}.
The values for these are generated for each form, based on generally followed guidelines for such fields. For example, password fields should consist of at least one uppercase letter, one lowercase letter, and special characters.

When the fuzzed data is ready, the Fuzzer constructs the appropriate HTTP request (GET or POST) and sends the HTTP request to the URL that was generated by the \Email Form Retriever (Section~\ref{Comp:EMFR}). 


\subsubsection{Injection Verification}
\label{Comp:EMA}
The Injection Verification module checks for the presence of injected data in the received \emails. This module works on the \emails received and stored by our Postfix server, and, depending on the user account that received the \email, it performs different functions.

\noindent\textbf{Analyzing regular e-mail.}
\sloppy
`Regular \email' refers to the \emails received by account {\lstinline{reguser#@example.com}} that were sent due to injecting the regular, non-malicious, payload (discussed in Section~\ref{Comp:Fuzzer}). The objective of the analysis on this \email is identify if the input fields that we injected with data appear on the resulting \email, and if so, which fields appear where.

To find this, we parse each received \email, and check whether \emph{any} of the fields we injected with data appear in either the headers or body of the \email. These could be fields such as name, username, age, etc. If they do, we add them to the list of fields that can \emph{potentially} result in an \ehi vulnerability for the given \email. 

We then pass on this information back to the Fuzzer pipeline, along with the vulnerable form, where these fields are \emph{also} fuzzed along with the \email fields to check for the presence of \ehi.

\noindent\textbf{Analyzing \email with payloads}
The ``\emails with payloads'' refers to \emails received by either the {\lstinline{nuser#@example.com}} or {\lstinline{maluser#@example.com}} accounts. These \emails could only be received as a result of injecting the malicious payloads that were discussed in Section~\ref{Comp:Fuzzer}. 

\noindent\textbf{Detecting injected \texttt{bcc} headers}
As discussed in the payloads Section~(\ref{Comp:Fuzzer}), the payloads were crafted such that the \emails received by the \texttt{maluser} account directly indicate the presence of the injected \texttt{bcc} field. 

\label{analyze:detect_x_check}
\noindent\textbf{Detecting injected \texttt{x-check} headers} \Emails
not received by the \texttt{maluser} account but by the \texttt{nuser}
account constitute a special category of \emails. These \emails could
have been generated due to two reasons: (1) The web application
performed sanitization and stripped out the \texttt{bcc}
part of the payload, thereby sending \emails only to the
\texttt{nuser} account. These \emails then act as proof that the
vulnerability was not found on the given URL. (2) The \texttt{bcc}
header can be ignored for other reasons (e.g.,\ Python's default
behavior when it encounters duplicate headers). In this case, we check
if the \email contains the custom header \texttt{x-check}. If it does,
then this is a successful exploit of the vulnerability.
