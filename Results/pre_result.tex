From our extensive crawl of the web, we were able to gather the data
shown in Table~\ref{tab:data}. We ran the system for 76 days, during which our system crawled \urls unique URLs,
and found a total of \forms\ forms from \uniqueforms\ unique domains. Out of these forms, our system
found \emailforms\ forms that contained an \email field, from \uniqueemailforms\ unique domains.
Table~\ref{tab:fuzzed_data} shows the quantity of \emails we received for the benign and malicious payloads. 
\begin{table}[tbp]
	\centering
	\scriptsize
	\begin{tabular}{|c|c|}
		\hline
		\multicolumn{1}{|c|}{\textbf{Type of Data}} &
		\multicolumn{1}{c|}{\textbf{Quantity}}\\
		\hline
		URLs Crawled & \urls \\
		\hline
		Total Forms found & \forms \\
		\hline
		Forms with E-Mail Fields & \emailforms \\
		\hline
	\end{tabular}
	\caption[\titlecap{Collected data}]{The data collected for our
      project.}
    \vspace{-5ex}
	\label{tab:data}
\end{table}

%\begin{table}[tbp]
	\centering
	\scriptsize
	\begin{tabular}{|c|c|c|}
		\hline
		\multicolumn{1}{|c|}{\textbf{Type of fuzzing}} &
		\multicolumn{1}{c|}{\textbf{Forms fuzzed}} &
		\multicolumn{1}{c|}{\textbf{E-Mails received}}\\
		\hline
		Regular payload & \fuzzed & \recd \\
		\hline
		Malicious payload & \malfuzzed & \success \\
		\hline
	\end{tabular}
	\caption[\titlecap{Fuzzed data}]{The data we fuzzed and the e-mails we received.}
    \vspace{-5ex}    
	\label{tab:fuzzed_data}
\end{table}


\noindent\textbf{\Email received from forms.} The \emails that we
received can be categorized into two categories. (1) \Emails due to
regular payload: This represents the total number of web applications
that sent \emails to us. This indicates that we were able to
successfully submit the forms on these sites to trigger the web
application to send an \email. (2) \Emails due to malicious payload:
Once we receive an \email from a web application due to the regular
payload, we fuzz those forms with the malicious payloads. This field
represents the total number of unique URLs that contain an \ehi
vulnerability.

