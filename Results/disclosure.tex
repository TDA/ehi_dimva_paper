\subsection{Responsible Disclosure of Discovered Vulnerabilities}

After we discovered an \ehi vulnerability on a particular website, we attempted to notify the developers of the vulnerable web application, along with a brief description of the vulnerability.
We chose to \email the following mailboxes, following the rules
specified in RFC~2142~\cite{rfc2142}: \texttt{security@domain.com},
\texttt{admin@domain.com}, and \texttt{webmaster@domain.com}
 
Out of the \domains\ vulnerable domains found, only
\emailedDefaultmailbox websites had the mailboxes able to receive
\emails. For the remaining domains, we used the
\texttt{whois}~\cite{whois} data to find the contact details of the
owner and then \emailed them. We received \responses developer
responses, confirming \confirmed discovered vulnerabilities. Four of
the developers fixed the vulnerability on their website.

From our research, it is clear that \ehi is quite widespread as a
vulnerability, appearing on \successDelta\ of forms that we were able
to perform automated attacks on. This value acts as a \emph{lower
  bound} for prevalence of \ehi vulnerability, and can quite easily be
larger if the attacks were broader, crafted for the individual web
application, and less automated.


%% \begin{table}[tbp]
%% \centering
%% \scriptsize
%% \begin{tabular}{|c|c|c|}
%% 	\hline
%% 	\multicolumn{1}{|p{2cm}}{\centering \textbf{Notified websites}} &
%% 	\multicolumn{1}{|p{2cm}|}{\centering \textbf{Developer Responses}} &
%% 	\multicolumn{1}{p{2cm}|}{\centering \textbf{Confirmed discoveries}}\\
%% 	\hline
%% 	\domains\ & \responses & \confirmed \\
%% 	\hline
%% \end{tabular}
%% 	\caption[\titlecap{}]{Responsible disclosure of the discovered vulnerabilities to developers and the number of received responses.}
%% 	\label{tab:devresp}
%% \end{table}

% Adam: did we get updated notifications? - updated. DONE.
