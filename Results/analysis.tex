\vspace{-2.5ex}
\subsection{Analysis of the Received \Email Data}
\vspace{-2.5ex}
During our analysis of the received \emails, we found that the \emails that we received belonged to three categories:

(1) \Emails with the \texttt{bcc} header successfully injected. This form
of injection was our initial objective, and we found
% Adam: Sai, why isn't this number a command? - DONE.
\ehibcc such \emails in our received \emails. This validates that the
web applications that sent these \emails are vulnerable to \ehi.
	
(2) \Emails with the \texttt{to} header successfully injected. We
discovered an unintended vulnerability class during our analysis,
which we call \texttt{To~header injection}. These injections reflect
the ability to inject any number of \email addresses into the
\texttt{to} field of the SMTP message while being unable to inject any
other header into the \emails. We found \ehito such \emails in our
received \emails. We attribute this behavior to inconsistent
sanitization by the application.
   
While not allowing us complete control over content of the \emails
sent, \texttt{To header injection} makes it possible to append any
number of \email addresses, thereby enabling us to leak information or
perform DoS (Denial of Service) attacks against the web application.
	
(3) \Emails with the \texttt{x-check} header successfully injected. The
third category of \emails received were \emails with the
\texttt{x-check} header injected. As discussed in
Section~\ref{analyze:detect_x_check}, we can differentiate between
unsuccessful attempts and successful attempts by injecting the
additional header and checking whether headers other than the
\texttt{bcc} header can be injected into the generated \email.
\ehixcheck \emails were received with the \texttt{x-check} header
injected.

% Adam: Sai, we need to put the actual numbers in the text. We cannot count on the readers to look at the table (of course, we should use the commands
We list each category and the number of \emails received by that
category in Table~\ref{tab:analysis}. We explain the combination of
these header injections (4-7) as follows:

%\begin{table}[tbp]
\begin{wraptable}{r}{6.5cm}
	\centering
	\scriptsize
    \vspace{-4ex}
	\begin{tabular}{|l|p{1cm}|}
		\hline
		\textbf{Type of Injection} & \textbf{\# e-mails}\\
		\hline
		E-Mail Header Injections with \texttt{bcc} header & \ehibcc \\
		\hline
		E-Mail Header Injections with \texttt{x-check} header & \ehixcheck \\
		\hline
		\texttt{To header} injections alone & \ehito \\
		\hline
		Injections with both \texttt{bcc} and \texttt{x-check} headers & \ehibccxcheck \\
		\hline
		Both \texttt{To header} injections and x-check headers &
		\ehitoxcheck \\
		\hline
		\texttt{x-check} headers found in \texttt{nuser} e-mails & \ehinuserxcheck \\
		\hline
		Unique \texttt{x-check} headers found in \texttt{nuser} e-mails & \ehiuniquenuserxcheck \\
		\hline
		Total successful injections (1 + 3 + 7) & \success \\
		
		\hline
	\end{tabular}
	\caption[\titlecap{Analysis of the data}]{Classification of the e-mails that we received into broad categories of the vulnerability.}
    \vspace{-5ex}
	\label{tab:analysis}
    %\end{table}
\end{wraptable}


\Email Header Injections with both \texttt{bcc} and \texttt{x-check}
headers: These represent the scenario where an attacker can inject
multiple headers into the \emails. We can see that 54\% of the
received \texttt{bcc} header injected \emails are also susceptible to
being injected with additional headers.
      % Adam: What is the exact percentage? We should use that rather than an approximate number - DONE.
	
Both \texttt{To} header injections and \texttt{x-check} headers: This
combination shows us that in addition to being able to inject into the
\texttt{To} fields, we injected additional headers into the \email. It
is not clear what causes this behavior; however, these can be
exploited to achieve the same result as a regular \ehi.
	
Total \texttt{x-check} headers and unique \texttt{x-check} headers
found in \texttt{nuser} \emails: We found a total of \ehinuserxcheck
\emails in the \texttt{nuser} account. Out of these,
\ehiuniquenuserxcheck had unique form ids that were \emph{not} already
found in the \texttt{maluser} account. We attribute these \emails to
(probably) being sent by a web application that was built with Python
or another language having a similar behavior with respect to ignoring
duplicate headers while constructing an \email, thus appending the
\texttt{x-check} header and \emph{not} the \texttt{bcc} header.
      % Adam: What is the actual behavior? This is confusing. - Changed. DONE.
	
Total successful injections: This represents the total number of
successful injections. This includes the \ehi with \texttt{bcc}
header\,(1), \texttt{To} header injections\,(3), and Unique
\texttt{x-check} headers found in \texttt{nuser} \emails\,(7). A total
of \success vulnerabilities were found by our system.

      % Adam: we need the actual #s here too. These are the core results of our analysis, they need to be in the text. - DONE.
