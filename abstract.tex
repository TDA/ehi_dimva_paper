
\begin{abstract}
	\ehi vulnerability is a class of vulnerability that can occur in web applications that use user input to construct \email messages. \ehi is possible when the script fails to check for the presence of \email headers in user input. We discovered that the vulnerability exists in the built-in \email functionality of the popular languages PHP, Java, Python, and Ruby. With the proper injection string, this vulnerability can be exploited to allow an attacker to inject additional headers, modify existing headers, and alter the content of the \email.

	While \ehi vulnerabilities are known by the community, and some commercial vulnerability scanners claim to discover \ehi vulnerabilities, they have never been studied by the academic community. This paper presents a scalable mechanism to automatically detect \ehi vulnerabilities and uses this mechanism to quantify the prevalence of \ehi vulnerabilities on the web. Using a black-box testing approach, the system crawled \urls URLs to identify web pages which contained form fields. \forms such forms were found by the system, of which \emailforms forms contained \email fields. We then fuzzed \fuzzed forms to see if they would send us an \email, and \recd forms sent us an \email. Of these, \malfuzzed forms were tested with \ehi payloads and, of these, we found \success vulnerable URLs across \domains domains. Then, to measure if \ehi vulnerabilities are actively being exploited to create a spamming platform, we found evidence that \ipsblacklist IPs that were vulnerable to \ehi are on spamming blacklists. We also performed a virus check on the received \emails and found 265 domains to be sending malicious content. This work shows that \ehi vulnerabilities are widespread and deserve future research attention.
\end{abstract}
