\vspace{-2.5ex}
\begin{abstract}
	\ehi vulnerability is a class of vulnerability that can occur in
    web applications that use user input to construct \email messages.
    \ehi vulnerabilities exist in the built-in \email functionality of
    the popular languages PHP, Java, Python, and Ruby. With the proper
    injection string, this vulnerability can be exploited to allow an
    attacker to inject additional headers, modify existing headers,
    and alter the content of the \email.

	While \ehi vulnerabilities are known to the community, and some
    commercial vulnerability scanners claim to discover \ehi
    vulnerabilities, they have never been studied by the academic
    community. This paper presents a scalable mechanism to
    automatically detect \ehi vulnerabilities and uses this mechanism
    to quantify the prevalence of \ehi vulnerabilities on the web.
    From crawling \urls URLs, we found \success vulnerable URLs across
    \domains domains. 135 of these domains are in the Alexa top 1
    million, and five of them ar ein the top 20,000. 137 of the
    vulnerable domains are using anti-spoofing mechanisms such as
    DIKM, SPF, or DMARC, and \ehi renders this protection useless.
    This work shows that \ehi vulnerabilities are widespread and
    deserve future research attention.

\end{abstract}
